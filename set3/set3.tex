\documentclass[12pt]{article}

\usepackage[utf8]{inputenc}
\usepackage[margin=0.6in]{geometry}

\usepackage{amsmath}
\usepackage{graphicx}
\usepackage{booktabs}

\newcommand{\D}[2]{\frac{\mathrm{d}#1}{\mathrm{d}#2}}
\newcommand{\pfrac}[2]{\left(\frac{#1}{#2} \right)}
\newcommand{\ex}[1]{\left\langle#1\right\rangle}

\newcommand{\pdiff}[2]{\frac{\partial#1}{\partial#2}}

\begin{document}

\section{Relative Velocity of Maxwell Boltzmann Gases}

Let's consider the joint probability for the velocities of one particle from each population:

\[ p(v_1,v_2) = \pfrac{\sqrt{m_1 m_2}}{2\pi kT}^3 e^{-(m_1 v_1^2 + m_2 v_2^2) / 2kT} d^3 v_1 d^3 v_2
\]

We can change variables to their center of mass velocity, \(V\) and relative velocity, \(v\), so that now

\begin{align*}
v_1 &= V + \frac{m_2}{M}v\\
v_2 &= V - \frac{m_1}{M}v\\
 p(v_1,v_2) &= \pfrac{\sqrt{m_1 m_2}}{2\pi kT}^3 e^{-(MV^2 + \mu v^2) / 2kT} d^3 v_1 d^3 v_2
\end{align*}

where \(M = m_1 + m_2\) and \(\mu\) is the reduced mass. Next we need to find the Jacobian Determinant of this transformation. Notice that it can be written in block matrix form as

\[ J = \left|  \begin{array}{cc} 
	\pdiff{v_1}{V} & \pdiff{v_2}{V} \\[11pt]
	\pdiff{v_1}{v} & \pdiff{v_2}{v}

\end{array} \right|
\]

where
\[ \pdiff{v_1}{V} = \left(\begin{array}{ccc}
	\pdiff{v_{1x}}{V_x} & \pdiff{v_{1y}}{V_x} & \pdiff{v_{1z}}{V_x} \\
	\pdiff{v_{1x}}{V_y} & \ddots &  \\
	\vdots & & \ddots
	
\end{array}\right)
\]\\

and so on. We can evaluate each block to get \\

\[ J = \left|  \begin{array}{cc} 
	I & I \\[11pt]
	\frac{m_2}{M}I & -\frac{m_1}{M}I 
\end{array} \right|
\]\\

Apparently, the algebra of block matrix determinants is fairly complicated in the general case, but there's a nice result for \(2\times 2\) matrices of square blocks:

\[ J = \left|  -\frac{m_1}{M}I^2 - \frac{m_2}{M}I^2 \right| = \left| -I\right| = (-1)^3 = -1
\]\\

So \(d^3 v_1 d^3 v_2 = d^3 V d^3 v\) and we can write

\[ p(V,v) = \pfrac{\sqrt{m_1 m_2}}{2\pi kT}^3 e^{-(MV^2 + \mu v^2) / 2kT} d^3 V d^3 v
\]

We can get just \(p(v)\) now by marginalizing over \(V\):

\begin{align*}
p(v) &= \int_V  p(V,v) d^3 V d^3 v \\[11pt]
&=\pfrac{\sqrt{m_1 m_2}}{2\pi kT}^3 \iiint dV_x dV_y dV_z e^{-\mu v^2 / 2kT} e^{-MV_x^2/2kT}e^{-MV_x^2/2kT}e^{-MV_x^2/2kT} d^3 v \\[11pt]
&=\pfrac{\sqrt{m_1 m_2}}{2\pi kT}^3  \pfrac{2\pi kT}{M}^{3/2} e^{-\mu v^2 / 2kT} d^3v \\[11pt]
&= \pfrac{\mu}{2\pi kT}^{3/2} e^{-\mu v^2 / 2kT} d^3v
\end{align*}

\section{Nuclear Reaction Chains and Abundances}

\subsection{}

\newcommand{\nc}[2]{n_{^{#1}\mathrm{#2}}}
\newcommand{\dnc}[2]{\dot{n}_{^{#1}\mathrm{#2}}}

Let \(r_i\) be the reaction rate for the \(i^{\mathrm{ith}}\) reaction in the PPI chain.

\begin{align*}
r_1 &= \frac{1}{2}\lambda_1 \nc{1}{H}^2 \\
r_2 &= \lambda_2 \nc{1}{H}\nc{2}{H} \\
r_3 &= \frac{1}{2}\lambda_3 \nc{3}{He}^2
\end{align*}

We have

\begin{align*}
\dnc{1}{H} &= -2r_1 - r_2 + 2r_3 \\
\dnc{2}{H} &= r_1 - r_2\\
\dnc{3}{He} &= -2r_3 + r_2 \\
\dnc{4}{He} &= r_3
\end{align*}

\subsection{}

We can take \(r_1 = r_2\) to find \(  \frac{\nc{2}{H}}{\nc{1}{H}} = \frac{\lambda_1}{2\lambda_2} \).

\section{He flash}

Keeping density and helium concentration constant, we're left with a pretty simple differential equation to step through. This is implemented in the attached notebook with a simple euler integrator.

\section{•}

Following the prescription, we take \(g_i = 1\) and solve for \(\mu_i\):

\[ \mu_i = m_i c^2 + kT \log{n_i} - \frac{3}{2}kT\log{\frac{m_i kT}{2\pi\hbar^2}}
\]


\newcommand{\muhe}{\mu_{\mathrm{He}}}
\newcommand{\muni}{\mu_{\mathrm{Ni}}}
\newcommand{\mhe}{m_{\mathrm{He}}}
\newcommand{\mni}{m_{\mathrm{Ni}}}
\newcommand{\nhe}{n_{\mathrm{He}}}
\newcommand{\nni}{n_{\mathrm{Ni}}}

Setting \(14\muhe = \muni\),
\[
\frac{\nhe^{14}}{\nni} = \pfrac{\mni kT}{2\pi\hbar^2}^{-\frac{3}{2}}\pfrac{\mhe kT}{2\pi\hbar^2}^{21} e^{(\mni-14\mhe)c^2/kT}
\]



\end{document}